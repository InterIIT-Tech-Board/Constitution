%
%
% Law Review Article Template
% A "universal" template 
% 
%

\documentclass[letterpaper,12pt,twoside]{article}

\usepackage[utf8]{inputenc}
\usepackage{mathptmx}
\usepackage{lipsum} % You can remove this package if you do not want to use filler text.
\usepackage{titlesec}
\usepackage{ragged2e}
\usepackage{fancyhdr}
\usepackage[all]{nowidow}
\usepackage{url}
\usepackage{tocloft}
\usepackage[headheight=110pt]{geometry}

%------
\makeatletter
\renewcommand\normalsize{\@setfontsize\normalsize{12}{14}}
\renewcommand\small{\@setfontsize\small{10}{12}}
\renewcommand\scriptsize{\@setfontsize\scriptsize{7}{8}}
\renewcommand\tiny{\@setfontsize\tiny{5}{6}}
\renewcommand\large{\@setfontsize\large{13}{15}}
\renewcommand\Large{\@setfontsize\Large{14}{16}}
\renewcommand{\thesection}{\Roman{section}}
\renewcommand{\thesubsection}{\Alph{subsection}}
\renewcommand{\thesubsubsection}{\arabic{subsubsection}}
\renewcommand{\theparagraph}{\alph{paragraph}}
\renewcommand{\thesubparagraph}{\roman{subparagraph}}
%------
\titleformat{\section}[block]{\normalfont\normalsize\scshape\center}{\thesection.}{1em}{}
\titleformat{\subsection}[block]{\normalfont\itshape\center}{\thesubsection.}{1em}{}
\titleformat{\subsubsection}{\normalfont}{\thesubsubsection.}{1em}{}
\titleformat{\paragraph}{\normalfont}{\theparagraph.}{1em}{}
\titleformat{\subparagraph}{\normalfont\small}{\thesubparagraph.}{1em}{}
\titlespacing{\subparagraph}{0pt}{12pt}{12pt}
\makeatother

\renewcommand\contentsname{\hspace*{\fill}\normalfont\normalsize\scshape Table of Contents\hspace*{\fill}}   
\renewcommand\cftsecfont{\normalfont\normalsize\scshape}
\renewcommand\cftsubsecfont{\normalfont\normalsize\itshape}
\renewcommand\cftsubsubsecfont{\normalfont\normalsize}
\renewcommand\cftparafont{\normalfont\normalsize}
\renewcommand\cftsubparafont{\normalfont\small}
\renewcommand\cftsecpagefont{\normalfont\normalsize}
\renewcommand\abstractname{\normalfont\normalsize\scshape Abstract}

\setcounter{secnumdepth}{5}
\setcounter{tocdepth}{5}

\cftsetindents{section}{3em}{3em}
\cftsetindents{subsection}{4em}{3em}
\cftsetindents{subsubsection}{5em}{3em}
\cftsetindents{paragraph}{6em}{3em}
\cftsetindents{subparagraph}{7em}{3em}

\fancypagestyle{mypagestyle}{%
  \fancyhf{}% Clear header/footer
  \fancyhead[OC]{\uppercase{Title Of The Article Here}}% Title on Odd pages, center.
  \fancyhead[OL]{16-Oct-01]}
  \fancyhead[OR]{\thepage}
  \fancyhead[EC]{\uppercase{Title Of The Article Here}}% Title on Even page, center.
  \fancyhead[EL]{\thepage}
  \fancyhead[ER]{[16-Oct-01}
  \fancyfoot[O]{}
  \fancyfoot[E]{}
  \renewcommand{\headrulewidth}{0pt}% Removes header rule
}
\pagestyle{mypagestyle}

\title{\Huge{\uppercase{\textbf{Constitution of the InterIIT Tech Meet}}}}
% \author{\large\emph{Name of Author}\thanks{Author affiliation.}}
%-- If you have more than one author and they have different affiliations, put the percent sign before
%-- the \author command above and remove it from the start of the \author command below.

%\author{\large\emph{Name of Author1}\thanks{Author affiliation1.} \and \large\emph{Name of Author2}\thanks{Author affiliation2.} \and \large\emph{Name of Author3}\thanks{Author affiliation3.}}

%--
\date{} % Leave blank for no date. If you fill in the date it will appear below the author's name	

\begin{document}
\urlstyle{same}

\maketitle
\newpage
\thispagestyle{empty}

	\begin{abstract}
	This is a generic template for a law review article. It is modeled after the Word law review article template that Eugene Volokh created. You can get to Professor Volokh's template here: \url{http://www2.law.ucla.edu/volokh/writing/}. This template, in addition to converting the coding from Word to \LaTeX,~adds some features. It includes the option of having multiple authors, using a table of contents, and including an abstract. The code indicates how to modify the template for an article with multiple authors and how to remove the table of contents, the abstract, or both. While there have been a few attempts to create a package that will automatically format citations using the Bluebook style, they have not gone very far or been very successful. So, you are left on your own. I have given examples in the text of how to use \LaTeX~codes to achieve the Bluebook style, which may help guide you. These should help you with most of the typesetting styles you need, and you can use the Bluebook to determine the particular requirements for a cite. 
	\end{abstract}

\tableofcontents % Delete this line and the next line and the next two lines if you do not want a table of contents
\vspace{14pt}
\clearpage

\addcontentsline{toc}{section}{Introduction} % Delete if no Table of Contents
\section*{Introduction}

\lipsum[1-2] % This command inserts filler text. Replace it with your text.

\section{First Level Heading}

\lipsum[1]

	\begin{quote}
		\lipsum[2]
	\end{quote}

\subsection{Second-Level Heading}

\lipsum[3]

The following list includes some examples of citations formatted according to the Bluebook style. These examples demonstrated how to use \LaTeX~typesetting codes to achieve the Bluebook style, but are not intended to explain how you should cite properly using the Bluebook style.

	\begin{itemize}
		\item Case: Jackson v. Metropolitan Edison Co., 348 F. Supp. 954, 956--58 (M.D. Pa. 1972), \emph{aff'd}, 483 F.2d 754 (3d Cir. 1973), \emph{aff'd}, 419 U.S. 345 (1974)
		\item Constitution: \textsc{N.M. Const.} art. IV. \S~7
		\item Statute: Administrative Procedure Act \S~6, 5 U.S.C. \S 555 (1982)
		\item Legislative Materials: H.R. 3055, 94th Cong., 2d Sess. \S~2, 122 \textsc{Cong. Rec.} 16,870 (1976)
		\item Book: 2 \textsc{F. Pollock \& F. Maitland, The History of English Law} 201--14 (1895)
		\item Work not formally printed: H. Wechsler, Remarks at the Meeting of the Bar of the Supreme Court of the United States in Memory of Chief Justice Stone 5 (Nov. 12, 1947) (available in Columbia Law School Library)
		\item Periodical: Rees, \emph{Legislative Jurisdiction}, 78 \textsc{Colum. L. Rev.} 1587, 1591--94 (1978)
		\item	Newspaper: Boston Globe, Oct. 14, 1954, at 6, col. 1
		\item Treaty: Parcel Post Agreement, June 3--14, 1951, United States-Gold Coast Colony, art. IV, 2 U.S.T. 1859, 1862, T.I.A.S. No. 2322, at 4
	\end{itemize}

This is a sentence with a footnote at the end that cites a case, a journal article, and a book, and includes a block quote.\footnote{\emph{See, e.g.,} Mapp v. Ohio, 367 U.S. 643 (1961); J.J. Prescott et al., \emph{Understanding Noncompetition Agreements: The 2014 Noncompete Survey Project}, 2016 \textsc{Mich. St. L. Rev.} 369 (2016); \textsc{Kenneth Culp Davis, Police Discretion 98--120 (1975)}. This is the block quote:
	\begin{quote}
		\lipsum[4]
	\end{quote}
	\emph{Id}. at 118.}

\subsubsection{Third-level heading}

\lipsum[5]

\paragraph{Fourth-level heading}

\noindent\lipsum[6]

\subparagraph{Fifth-level heading}

\noindent\lipsum[7]

\addcontentsline{toc}{section}{Conclusion} % Delete if no Table of Contents
\section*{Conclusion}

\lipsum[8]

\end{document}
