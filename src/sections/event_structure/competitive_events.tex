% Content for Competitive Events
\paragraph{}
The \st{Host IIT} Organising Team will ideate domains for the competitions. These problem statements will be released on the official website as per the schedule. 

\paragraph{Generic guidelines for Problem Statements} 
The competitive elements of some events are included to provide small incentives and encourage participants to deliver despite the challenges. This is to fuel innovation under given constraints.

\paragraph{}
Competition problem statements should be so chosen that they are in line with our vision and mission. Some of the domains (non-exhaustive) which could be considered are listed below:

\begin{enumerate}
    \item Sustainability - energy, waste management, water conservation etc.,
    \item Technology for healthcare,
    \item Technology for the upliftment of agriculture,
    \item Education technology,
    \item Transportation technology,
    \item Technologies enhancing indigenous manufacturing capabilities,
    \item Techno-finance and Techno-design based problems.
\end{enumerate}

\subsubsection{Categorisation of Competitive Events}
% Content for Event Categories
\paragraph{}
The \st{Host IIT} Organising Team shall propose categorisation of competition based on the following criteria:
\begin{enumerate}
    \item expected number size of the team working on problem,
    \item amount of time to be devoted towards problem-solving prior to attending the event
    \item whether competition demands a hardware proof of concept/demonstration or involves the implementation of the solution.
    \item monetary or non-monetary resources involved in the project.
\end{enumerate}

\paragraph{}
The \st{Host IIT} Organising Team shall be presenting the categorisation adopted by them during first tech board meeting. \st{Host IIT} Organising Team shall release the competition categorisation to be adopted by them at least one week before the board meet on the official forum of communication and shall try to incorporate changes (if any). In case the \st{Host IIT} Organising Team fails to get a majority in voting for their proposed categorisation, they shall follow the competition categorisation structure as presented below.

\paragraph{Competitions have been categorised as follows:}
\begin{enumerate}
    \item 
    \textbf{High-Prep Competition}\\
    These are the competitions that involve proof-of-concept demonstration, implementation etc., that happen during the meet. This requires extensive preparation of any prototypes, submissions, etc., from the contingent as per the problem statement. The problem-solving will require sustained efforts of 4-10 weeks or more with weekly input of 10-30 hours. The typical team size required for this will be more than 6. A significant amount of prototyping costs/resources will be involved in high-prep competition
    \item 
    \textbf{Mid-Prep Competition}\\
    These are the competitions that involve demonstration or presentations that happen during the meet. This may require the preparation of some prototypes, submissions, etc., from the contingents as per the problem statement. The problem-solving will require a sustained effort of anywhere between 2-4 weeks with weekly input of 8-20 hours. The typical team size required for this will be between 3-8 people.
    \item 
    \textbf{Low-Prep Competition}\\
    These are the competitions that involve presentations that happen during the meet. This may require the preparation of some submissions, etc., from the contingents as per the problem statement. The problem-solving will require a sustained effort of anywhere between 4-7 days. The typical team size required for this will be between 2-4 people.
    \item 
    \textbf{No-Prep Competition} (optional)\\
    These are the competitions that involve that require on-the-spot efforts with no prior preparation of any prototypes, submissions, etc., from the contingent. The typical team size required for this will be between 2-4 people.
\end{enumerate}

\paragraph{}
The event points of High-Prep competition must be greater than Mid-Prep and Low/No-Prep competition and Mid-Prep competition must greater than Low/No-Prep competition.